\documentclass[12pt,a4paper]{article}

\usepackage[utf8]{inputenc}
\usepackage[T1]{fontenc}
\usepackage[english]{babel}
\usepackage{geometry}
\geometry{margin=2.5cm}
\usepackage{hyperref}

\title{Outline of the Seminar Paper\\
\large  Community Detection in Bacterial and Viral Networks}
\author{Jonas Ziegler}
\date{\today}

\begin{document}

\maketitle

\section*{Proposed Structure of the Paper (15--20 pages)}

\begin{enumerate}

\item \textbf{Introduction (2--3 pages)}
\begin{enumerate}
  \item Motivation
  \begin{itemize}
    \item Why networks are important?
    \item Biological networks (protein interactions, gene regulation)
    \item Why community detection?
  \end{itemize}

  \item Definition of a Network
  \begin{itemize}
    \item Graph as a mathematical object
    \item Nodes and edges
    \item Adjacency list
    \item Adjacency matrix
    \item Directed vs. undirected networks
    \item Weighted vs. unweighted networks
    \item Weighted undireccted network: $G = (V, E, W)$
  \end{itemize}

  \item What can we do with Networks?
  \begin{itemize}
    \item Descriptive statistics
    \item Structural analysis
    \item Clustering and communities 
    \item Prediction of Future Community Behavior
  \end{itemize}
\end{enumerate}

\item \textbf{Network Summary Statistics (2--3 pages)}
\begin{enumerate}
  \item Basic Metrics
  \begin{itemize}
    \item Number of nodes
    \item Number of edges
    \item Density
    \item Average degree
  \end{itemize}

  \item Degree Distribution
  \begin{itemize}
    \item Scale-free networks
    \item Power-law behavior (if relevant)
  \end{itemize}

  \item Centrality Measures
  \begin{itemize}
    \item Degree centrality
    \item Betweenness centrality
    \begin{itemize}
      \item Node betweenness
      \item Edge betweenness
    \end{itemize}
    \item Closeness and betweenness (brief)
  \end{itemize}
\end{enumerate}

\item \textbf{Similarity Measures for Community Comparison (2 pages)}
\begin{enumerate}
  \item Adjusted Rand Index (ARI)
  \item Adjusted Mutual Information (AMI)
  \item Comparison of ARI and AMI
  \begin{itemize}
    \item When to use which measure
  \end{itemize}
\end{enumerate}

\item \textbf{Community Detection Methods (4--5 pages)}
\begin{enumerate}
  \item Modularity
  \begin{itemize}
    \item Basic idea
    \item Modularity formula:
    \[
    Q = \frac{1}{2m} \sum_{ij} (A_{ij} - P_{ij}) \delta(c_i, c_j)
    \]
  \end{itemize}

  \item Edge Betweenness (Girvan--Newman)
  \begin{itemize}
    \item Main focus of the paper
    \item Idea: edges with high betweenness separate communities
    \item Algorithm:
    \begin{enumerate}
      \item Compute edge betweenness
      \item Remove the strongest edge
      \item Repeat
    \end{enumerate}
    \item Advantages and disadvantages
  \end{itemize}

  \item Louvain Method
  \begin{itemize}
    \item Greedy optimization of modularity
    \item Very fast
    \item Hierarchical structure
    \item Limitations of Louvain
    \item Motivation for improvement -> Leiden
  \end{itemize}

  \item Leiden Method
  \begin{itemize}
    \item Improvement over Louvain
    \item Guarantees well-connected communities
    \item State-of-the-art method
  \end{itemize}
\end{enumerate}

\newpage
\item Data Description (2 pages)
\begin{enumerate}
  \item Data Source
  \begin{itemize}
    \item Bacterial protein similarity networks
    \item Viral protein similarity networks
  \end{itemize}

  \item Descriptive Comparison of Bac and Vir
  \begin{itemize}
    \item Size
    \item Density
    \item Degree distributions
  \end{itemize}

  \item Network Construction
  \begin{itemize}
    \item Nodes represent proteins
    \item Edges represent sequence similarity
    \item Weights: bit score
    \item bit score formula:
\[
S' = \frac{\lambda S - \ln K}{\ln 2}
\]
  \end{itemize}

  \item Preprocessing
  \begin{itemize}
    \item Filtering
    \item Removal of self-loops
    \item Largest connected component
  \end{itemize}
\end{enumerate}

\item Experimental Setup (2 pages)
\begin{enumerate}
  \item Implementation
  \begin{itemize}
    \item Python (NetworkX)
    \item \url{https://github.com/j0nascz/Paper_SMNCA.git}
  \end{itemize}

  \item Parameters
  \begin{itemize}
    \item Louvain: resolution parameter
    \item Leiden: resolution parameter
    \item Edge betweenness: number of cuts
  \end{itemize}

\item Evaluation Criteria
\begin{itemize}
    \item Pairwise comparison between methods and parameter settings
    \begin{itemize}
      \item Changing parameters(e.g resolution parameter) in Leiden and Lovain and comparing results with ARI and AMI
    \end{itemize}

\end{itemize}

\end{enumerate}


\item Results (3--4 pages)
\begin{enumerate}
  \item Method Comparison
  \begin{itemize}
    \item Tables: number of communities
    \item Modularity values
    \item Runtime
  \end{itemize}

  \newpage
  \item Hyperparameter Sensitivity Analysis
  \begin{itemize}
    \item Plots: resolution vs. number of communities
    \item ARI and AMI vs. resolution
  \end{itemize}

  \item Interpretation
  \begin{itemize}
    \item Stability of communities
  \end{itemize}
\end{enumerate}

\item Discussion (2 pages)
\begin{itemize}
    \item Why do Bac and Vir differ?
    \begin{itemize}
        \item Structural differences between the networks
        \item Effect on detected communities
    \end{itemize}
    \item Which method is more stable?
    \begin{itemize}
        \item Interpretation of stability based on ARI and AMI
        \item Intra-method vs. inter-method differences
    \end{itemize}
    \item Interpretation of ARI/AMI results
    \item Resolution limit problem
\end{itemize}

\item Conclusion (1 page)
\begin{itemize}
  \item Summary of findings
  \item Main insights
  \item Methodological implications
  \item Outlook and future work
\end{itemize}

\end{enumerate}

\end{document}
