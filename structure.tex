\documentclass[12pt,a4paper]{article}

\usepackage[utf8]{inputenc}
\usepackage[T1]{fontenc}
\usepackage[english]{babel}
\usepackage{geometry}
\geometry{margin=2.5cm}
\usepackage{hyperref}

\title{Outline of the Seminar Paper\\
\large  Community Detection in Bacterial and Viral Networks}
\author{Jonas Ziegler}
\date{\today}

\begin{document}

\maketitle

\section*{Proposed Structure of the Paper (15--20 pages)}

\begin{enumerate}

\item \textbf{Introduction (2--3 pages)}
\begin{enumerate}
  \item Motivation
  \begin{itemize}
    \item Why are networks important?
    \item Biological networks (protein interactions, gene regulation)
    \item Social networks
    \item Information networks
    \item Why community detection?
    \item Functional modules
    \item Organization of complex systems
  \end{itemize}

  \item Definition of a Network
  \begin{itemize}
    \item Graph as a mathematical object
    \item Nodes and edges
    \item Directed vs. undirected networks
    \item Weighted vs. unweighted networks
  \end{itemize}

  \item Mathematical Representation
  \begin{itemize}
    \item Adjacency list
    \item Adjacency matrix
    \item Formal definition: $G = (V, E)$
  \end{itemize}

  \item What Can We Do with Networks?
  \begin{itemize}
    \item Descriptive statistics
    \item Structural analysis
    \item Prediction
    \item Clustering and communities
  \end{itemize}
\end{enumerate}

\item \textbf{Network Summary Statistics (2--3 pages)}
\begin{enumerate}
  \item Basic Metrics
  \begin{itemize}
    \item Number of nodes
    \item Number of edges
    \item Density
    \item Average degree
  \end{itemize}

  \item Degree Distribution
  \begin{itemize}
    \item Scale-free networks
    \item Power-law behavior (if relevant)
  \end{itemize}

  \item Centrality Measures
  \begin{itemize}
    \item Degree centrality
    \item Edge betweenness
    \item Closeness and betweenness (brief)
  \end{itemize}
\end{enumerate}

\item \textbf{Similarity Measures for Community Comparison (2 pages)}
\begin{enumerate}
  \item Rand Index
  \begin{itemize}
    \item Definition and intuition
  \end{itemize}

  \item Adjusted Rand Index (ARI)
  \begin{itemize}
    \item Correction for chance
  \end{itemize}

  \item Adjusted Mutual Information (AMI)
  \begin{itemize}
    \item Information-theoretic perspective
  \end{itemize}

  \item Comparison of ARI and AMI
  \begin{itemize}
    \item When to use which measure
  \end{itemize}
\end{enumerate}

\item \textbf{Community Detection Methods (4--5 pages)}
\begin{enumerate}
  \item Modularity
  \begin{itemize}
    \item Basic idea
    \item Modularity formula:
    \[
    Q = \frac{1}{2m} \sum_{ij} (A_{ij} - P_{ij}) \delta(c_i, c_j)
    \]
  \end{itemize}

  \item Edge Betweenness (Girvan--Newman)
  \begin{itemize}
    \item Main focus of the paper
    \item Idea: edges with high betweenness separate communities
    \item Algorithm:
    \begin{enumerate}
      \item Compute edge betweenness
      \item Remove the strongest edge
      \item Repeat
    \end{enumerate}
    \item Advantages and disadvantages
  \end{itemize}

  \item Louvain Method
  \begin{itemize}
    \item Greedy optimization of modularity
    \item Very fast
    \item Hierarchical structure
  \end{itemize}

  \item Leiden Method
  \begin{itemize}
    \item Improvement over Louvain
    \item Guarantees well-connected communities
    \item State-of-the-art method
  \end{itemize}
\end{enumerate}

\item \textbf{Data Description (2 pages)}
\begin{enumerate}
  \item Data Source
  \begin{itemize}
    \item Bacterial network
    \item Viral network
    \item Protein homology networks
  \end{itemize}

  \item Network Structure
  \begin{itemize}
    \item Nodes represent proteins
    \item Edges represent sequence similarity
    \item Weights: bit score or e-value
  \end{itemize}

  \item Preprocessing
  \begin{itemize}
    \item Filtering
    \item Removal of self-loops
    \item Largest connected component
  \end{itemize}
\end{enumerate}

\item \textbf{Experimental Setup (2 pages)}
\begin{enumerate}
  \item Implementation
  \begin{itemize}
    \item Python (NetworkX)
    \item R (igraph)
  \end{itemize}

  \item Parameters
  \begin{itemize}
    \item Louvain: resolution parameter
    \item Leiden: resolution parameter
    \item Edge betweenness: number of cuts
  \end{itemize}

  \item Evaluation Criteria
  \begin{itemize}
    \item Number of communities
    \item Community sizes
    \item Modularity
    \item ARI
    \item AMI
  \end{itemize}
\end{enumerate}

\item \textbf{Results (3--4 pages)}
\begin{enumerate}
  \item Descriptive Comparison of Bac and Vir
  \begin{itemize}
    \item Size
    \item Density
    \item Degree distributions
  \end{itemize}

  \item Method Comparison
  \begin{itemize}
    \item Tables: number of communities
    \item Modularity values
    \item Runtime
  \end{itemize}

  \item Hyperparameter Sensitivity Analysis
  \begin{itemize}
    \item Plots: resolution vs. number of communities
    \item ARI and AMI vs. resolution
  \end{itemize}

  \item Interpretation
  \begin{itemize}
    \item Stability of communities
    \item Biological plausibility
  \end{itemize}
\end{enumerate}

\item \textbf{Discussion (2 pages)}
\begin{itemize}
  \item Why do Bac and Vir differ?
  \item Which method is more stable?
  \item When does modularity fail?
  \item Resolution limit problem
\end{itemize}

\item \textbf{Conclusion (1 page)}
\begin{itemize}
  \item Summary of findings
  \item Main insights
  \item Methodological implications
  \item Outlook and future work
\end{itemize}

\end{enumerate}

\end{document}
