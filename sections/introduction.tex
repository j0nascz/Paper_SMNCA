\section{Introduction}
\subsection{Motivation}
We live in a highly interconnected world where 
many physical, social, technological and 
biological systems consist of agents
or entities interacting with each other.
Examples include a virus being transmitted 
over social contact networks, global trade between 
countries, and the human brain. 
Any such system can be represented as a network 
by denoting the entities as nodes and the 
interactions between them as edges. 
This makes networks an important type of data 
spanning a remarkable variety of complex systems. 
It is therefore very important to have 
mathematically rigorous and practically 
useful methods for statistical analysis of networks. 
In this paper, we will focus on Community Detection in 
biological networks, in particular 
in bacterial and viral protein similarity networks. 

A protein-protein interaction network (PPI network), 
also known as the interactome,
is a type of biological network where a node represents 
a protein and an edge represents the physical 
interaction between each pair of proteins. 
These interactions are mapped using 
experimental techniques, 
such as yeast two-hybrid or mass spectrometry.
\textcite{newman2006} argues that community detection is 
one of the most important problems in the study 
of networks, and that it is 
particularly relevant for biological networks.


